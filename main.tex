\documentclass{article}
\usepackage[utf8]{inputenc}
\usepackage{graphicx}
\usepackage{verbatim}
\usepackage{listings}
\usepackage{amsmath}
\usepackage{tikz}
\usepackage{rotating}
\usetikzlibrary{positioning}
\usepackage{bussproofs}
\usepackage{turnstile}
\usepackage{stmaryrd}
\usepackage{caption}
\usepackage{subcaption}

\usepackage{cancel}
\newcommand{\lnec}{\Box}

\usepackage[edges]{forest}
\usepackage{amssymb}
\usepackage{comment}
\usepackage{minted}
\newcommand\mymapsto{\mathrel{\ooalign{$\rightarrow$\cr%
    \kern-.15ex\raise.275ex\hbox{\scalebox{1}[0.522]{$\mid$}}\cr}}}
\definecolor {processblue}{cmyk}{0.96,0,0,0}

\title{DM872 assignment 2}
\author{sagra16 \\(Valdemar Grange - 081097-2033) }
\date{June 2020}

\begin{document}

    \maketitle
    
    \clearpage

    \section*{Task 1}
    The (capacitated vehicle routing problem) CVRP can be formulated as a set-partitioning problem as follows.

    \begin{equation*}
        \begin{array}{ll@{}ll}
            \text{minimize}  & \displaystyle\sum_{r \in \Omega} c_r \theta_r& &\\
            \text{subject to}& \displaystyle\sum_{r \in \Omega} a_{i, r} \theta_r ,&   &\forall i \in N\\
                             & \displaystyle\sum \theta_r \leq m &&\\\\
                             & \displaystyle \theta_r  \in \{0,1\},  && \forall r \in \Omega
        \end{array}
    \end{equation*}
    
    Compared to the tasked problem we need a couple of things to complete the model.
    We implicitly only allow routes that satisfy the capacity limit by only allowing such sets in the problem.\\\\
    The "additional" aspect is the time windows, which we can model quite easily by the following constraint.
    \[
    \theta_r \, a_{i,r} \, q_{i,r} \leq l_i   , \,\,\,\,\,\,\,\,\, \forall i \in N, \, \forall r \in \Omega
    \]
    We let $q_{i,r}$ denote the time it takes to reach customer $i$ in route $r$.
    This can simply be implemented by.
\begin{minted}{python}
def q(i, r):
    accum = 0
    rRoutes =  Omega[r]
    routesBefore = rRoutes[:i - 1]
    for j in routesBefore:
        # Route from j - 1 to j
        travelCost = timeCost[i, j]
        # Account for waiting time if we are too early
        # The time when we are here (i) is accum + time for this travel step
        # If the difference from e_i, eg e_i - (travelCost + accum)
        # is positive, then we must wait for so long
        # such that the customer can be served.
        additionalCost = min(0, e_i - (travelCost + accum))
        accum = additionalCost + travelCost + accum
    return accum
\end{minted}
    Keep in mind that this will effectively create many constraints, as the amount of constraints is the number of customers times the number of routes.
    Which is by far the most "explosive" set of constraints.
 

\end{document}